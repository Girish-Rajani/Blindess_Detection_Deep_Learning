% Options for packages loaded elsewhere
\PassOptionsToPackage{unicode}{hyperref}
\PassOptionsToPackage{hyphens}{url}
%
\documentclass[
]{article}
\usepackage{lmodern}
\usepackage{amssymb,amsmath}
\usepackage{ifxetex,ifluatex}
\usepackage{tabularx}
\usepackage{caption}
\DeclareCaptionLabelFormat{table}{Table #2}
\usepackage[margin=1in]{geometry}
\ifnum 0\ifxetex 1\fi\ifluatex 1\fi=0 % if pdftex
  \usepackage[T1]{fontenc}
  \usepackage[utf8]{inputenc}
  \usepackage{textcomp} % provide euro and other symbols
\else % if luatex or xetex
  \usepackage{unicode-math}
  \defaultfontfeatures{Scale=MatchLowercase}
  \defaultfontfeatures[\rmfamily]{Ligatures=TeX,Scale=1}
\fi
% Use upquote if available, for straight quotes in verbatim environments
\IfFileExists{upquote.sty}{\usepackage{upquote}}{}
\IfFileExists{microtype.sty}{% use microtype if available
  \usepackage[]{microtype}
  \UseMicrotypeSet[protrusion]{basicmath} % disable protrusion for tt fonts
}{}
\makeatletter
\@ifundefined{KOMAClassName}{% if non-KOMA class
  \IfFileExists{parskip.sty}{%
    \usepackage{parskip}
  }{% else
    \setlength{\parindent}{0pt}
    \setlength{\parskip}{6pt plus 2pt minus 1pt}}
}{% if KOMA class
  \KOMAoptions{parskip=half}}
\makeatother
\usepackage{xcolor}
\IfFileExists{xurl.sty}{\usepackage{xurl}}{} % add URL line breaks if available
\IfFileExists{bookmark.sty}{\usepackage{bookmark}}{\usepackage{hyperref}}
\hypersetup{
  hidelinks,
  pdfcreator={LaTeX via pandoc}}
\urlstyle{same} % disable monospaced font for URLs
\usepackage{longtable,booktabs}
% Correct order of tables after \paragraph or \subparagraph
\usepackage{etoolbox}
\makeatletter
\patchcmd\longtable{\par}{\if@noskipsec\mbox{}\fi\par}{}{}
\makeatother
% Allow footnotes in longtable head/foot
\IfFileExists{footnotehyper.sty}{\usepackage{footnotehyper}}{\usepackage{footnote}}
\makesavenoteenv{longtable}
\usepackage{graphicx}
\makeatletter
\def\maxwidth{\ifdim\Gin@nat@width>\linewidth\linewidth\else\Gin@nat@width\fi}
\def\maxheight{\ifdim\Gin@nat@height>\textheight\textheight\else\Gin@nat@height\fi}
\makeatother
% Scale images if necessary, so that they will not overflow the page
% margins by default, and it is still possible to overwrite the defaults
% using explicit options in \includegraphics[width, height, ...]{}
\setkeys{Gin}{width=\maxwidth,height=\maxheight,keepaspectratio}
% Set default figure placement to htbp
\makeatletter
\def\fps@figure{htbp}
\makeatother
\setlength{\emergencystretch}{3em} % prevent overfull lines
\providecommand{\tightlist}{%
  \setlength{\itemsep}{0pt}\setlength{\parskip}{0pt}}
\setcounter{secnumdepth}{-\maxdimen} % remove section numbering
\ifluatex
  \usepackage{selnolig}  % disable illegal ligatures
\fi

\author{}
\date{}

\begin{document}

\begin{figure}[htbp]
\centerline{\includegraphics[width=1.5in,height=1.5in]{./Figure0.png}}
\label{fig}
\end{figure}

\vspace{3mm}
\centerline{\textbf{\LARGE ILLINOIS INSTITUTE OF TECHONOLOGY}\vspace{8mm}}
\centerline{\text{\Large CS 584 Machine Learning}\vspace{8mm}}
\centerline{\textbf{\LARGE Blindness Detection}\vspace{15mm}}
\centerline{\textbf{\textit{\large Girish Rajani-Bathija}}\vspace{4mm}}
\centerline{\textbf{\large grajanibathija@hawk.iit.edu}\vspace{4mm}}
\centerline{\textbf{\large A20503736}\vspace{15mm}}
\centerline{\textbf{\textit{\large Shriya Prasanna}}\vspace{4mm}}
\centerline{\textbf{\large sprasanna@hawk.iit.edu}\vspace{4mm}}
\centerline{\textbf{\large A20521733}\vspace{15mm}}
\centerline{\textbf{\textit{\large Bhavesh Rajesh Talreja}}\vspace{4mm}}
\centerline{\textbf{\large btalreja@hawk.iit.edu}\vspace{4mm}}
\centerline{\textbf{\large A20516822}\vspace{15mm}}
\centerline{\textbf{\textit{\large Prof. Yan Yan}}\vspace{8mm}}
\centerline{\text{\large Submission Date: $1^{\text{st}}$ May 2023}\vspace{8mm}}
\centerline{\textbf{\large IT IS DECLARED WITH MUTUAL AGREEMENT THAT EACH MEMBER HAS EQUALLY}}
\centerline{\textbf{\large CONTRIBUTED TO THE PROJECT}}
\newpage

%\includegraphics[width=5.5in,height=2.5in]{./Figure0.png}
%
%\textbf{ILLINOIS INSTITUTE OF TECHNOLOGY}
%
%CS 584 Machine Learning
%
%\textbf{Blindness Detection}
%
%\emph{\textbf{Girish Rajani-Bathija}}
%
%grajanibathija@hawk.iit.edu
%
%\textbf{A20503736}
%
%\emph{\textbf{Shriya Prasanna}}
%
%sprasanna@hawk.iit.edu
%
%\textbf{A20521733}
%
%\emph{\textbf{Bhavesh Rajesh Talreja}}
%
%btalreja@hawk.iit.edu
%
%\textbf{A20516822}
%
%\emph{\textbf{Prof. Yan Yan}}
%
%Submission Date: 1\textsuperscript{st} May 2023
%
%\textbf{IT IS DECLARED WITH MUTUAL AGREEMENT THAT EACH MEMBER HAS
%EQUALLY CONTRIBUTED TO THE PROJECT}
%
%\newpage

\hypertarget{introduction}{%
\subsection{\texorpdfstring{\textbf{1. Introduction}}{Introduction}}\label{introduction}}

Diabetic retinopathy is one of the most common causes of blindness in
people who have diabetes. Having high blood sugar from diabetes for a
prolonged period of time can lead to diabetic retinopathy and, if not
treated early, can lead to vision loss. For early detection of DR, there
needs to be a manual screening exam that can identify the risk and level
of someone developing this disease. However, in some rural areas, this
exam can be difficult to conduct due to a lack of access to
infrastructure and skilled professionals. Additionally, this manual
process of screening can be very time-consuming and labor-intensive. Due
to the ever-increasing demand for diabetic retinopathy screening, there
arises a need to automate such a process.

The goal of this project is to use algorithms in Machine Learning and
Deep Learning, such as Convolutional Neural Networks to extract features
from retina images to automate this process. This will be done by
performing multi-class image classification using the APTOS 2019
Blindness Detection dataset. Machine learning models allow us to speed
up the process of detecting DR early and make this service available to
those who may not have access to the infrastructure.

The APTOS 2019 Blindness Detection dataset available on Kaggle will be
used for this project. This real-world dataset consists of retina images
from multiple clinics (3,662 training images and 1,928 testing images)
in rural areas of India. Each training image has been carefully examined
and labeled as belonging to 1 of 5 classes - No DR, Mild DR, Moderate
DR, Severe DR, and Proliferative DR. However, the testing dataset is not
labeled as the goal is to label it.

\hypertarget{background-and-related-work}{%
\subsection{\texorpdfstring{\textbf{2. Background and Related
Work}}{Background and Related Work}}\label{background-and-related-work}}

Research conducted by Niloy, Sanaullah, Abu, and Abdullah {[}1{]} use
Ensemble Learning to perform early blindness detection using the retinal
images from the APTOS 2019 Blindness Detection dataset. In this study,
over 600 noisy images were detected and removed by manually examining
each image. Additionally, as part of the preprocessing phase, unwanted
information from the images was cropped, the images were all resized to
a uniform dimension, image augmentation was performed on the classes
that had fewer samples in order to obtain a more balanced dataset, tone
mapping the images, and finally omitting black corners. For training
purposes, an ET classifier which is an ensemble learning technique based
on decision trees and bagging was used for this problem. The study was
able to achieve an average accuracy of 91\% using this approach. For
this project, our group aims to implement some of the preprocessing
techniques done in this research paper and see how our Deep Learning
approach compares to that of the ET classifier.

The study conducted by Carson, Darvin, Margaret, and Tony {[}2{]} used
convolutional neural networks (CNNs) and various Transfer Learning
models such as VGG16, GoogLeNet, and AlexNet for image classification on
the APTOS 2019 Blindness Detection dataset. In this study, the
preprocessing phase involved copping the images, normalizing, and
performing color adjustment. The various models implemented in this
paper were able to achieve an accuracy of 57.2\%, 68.6\%, and 74.5\%.
For this project, our group aims to implement a custom CNN model as well
as a few of the transfer learning models used in this study, and the
results will be compared.

\hypertarget{methods}{%
\subsection{\texorpdfstring{\textbf{3. Methods}}{Methods}}\label{methods}}

\hypertarget{preprocessing}{%
\subsection{\texorpdfstring{\textbf{3.1
Preprocessing}}{3.1 Preprocessing}}\label{preprocessing}}

The retina images within this dataset required various preprocessing
techniques before using them in the Machine Learning models.

\hypertarget{trainvalidationtest-split}{%
\subsection{\texorpdfstring{\textbf{3.1.1 Train/Validation/Test
Split}}{3.1.1 Train/Validation/Test Split}}\label{trainvalidationtest-split}}

For this project, the validation set approach was selected, and so it
was necessary to split the training dataset consisting of 3,662 samples
into training set (2,462 samples) and validation set (1,200 samples).

Additionally, all of the training images were in one folder, but for
this multi-class classification problem, it was necessary to have all
images for each class in separate folders based on their label. For both
the training and validation sets, five folders were created (one for
each class), and the data were split into their respective classes using
the diagnosis feature within the train\_images.csv file. Now that the
training, validation, and test set have been split as needed, the images
had to be preprocessed.

\hypertarget{image-resizing}{%
\subsection{\texorpdfstring{\textbf{3.1.2 Image
Resizing}}{3.1.2 Image Resizing}}\label{image-resizing}}

It was observed that the dimensions of the images were different,
ranging from 474 x 358 pixels to 3388 x 2588 pixels. For an equal
comparison, it is crucial that all images are uniform in size and
dimension. To accomplish this, all images within the dataset were
resized to 200 x 200 pixels for uniformity. Sample images before and
after resizing can be observed in Figures 1 and 2 below.

\begin{figure}[h]
  \centering
  \includegraphics[width=6.5in,height=7.0in]{./Figure1.png}
  \caption{Showing a sample image from each class along with their
dimensions before resizing}
  \label{fig:figure1}
\end{figure}

\begin{figure}[h]
  \centering
  \includegraphics[width=6.5in,height=7.0in]{./Figure2.png}
  \caption{Showing a sample image from each class along with their
dimensions after resizing}
  \label{fig:figure2}
\end{figure}

\hypertarget{image-augmentation}{%
\subsection{\texorpdfstring{\textbf{3.1.3 Image
Augmentation}}{3.1.3 Image Augmentation}}\label{image-augmentation}}

The final step in the data preprocessing stage was data augmentation. It
was observed that the distribution of samples among the five classes
varied significantly. The table below shows this distribution which allows
us to conclude that the dataset is imbalanced.

\begin{figure}[hbt!]
  \centering
  \includegraphics[width=5.5in,height=5.0in]{./Figure3.png}
  \caption{Showing the distribution of samples for each of the 5 classes
before augmentation}
  \label{fig:figure3}
\end{figure}

As shown below, it can be seen that in both the training and validation
set, classes 1, 3, and 4 have significantly fewer samples than classes 0
and 2. Therefore, to reduce the effect of bias in training, data
augmentation (rotation) was performed on the classes that had fewer
samples to increase the number of samples within those classes and
achieve a more balanced distribution.

\begin{figure}[hbt!]
  \centering
  \includegraphics[width=5.5in,height=5.0in]{./Figure4.png}
  \caption{Showing the distribution of samples for each of the 5 classes
after augmentation}
  \label{fig:figure3}
\end{figure}

\newpage

\hypertarget{data-augmentation}{%
\subsection{\texorpdfstring{\textbf{3.2 Data
Augmentation}}{3.2 Data Augmentation}}\label{data-augmentation}}

Although data augmentation had been performed within the preprocessing
stage as explained above, that was only on 3 out of the 5 classes with
the purpose of achieving a more balanced distribution of samples within
each class to mitigate any effect of bias during training.

During training, each model that is created will be trained on training
data with and without data augmentation to see how increasing all of the
samples affects the model's ability to learn and generalize better. In
this stage, when performing data augmentation, all samples within the
training set will be augmented, and various methods will be used, such
as rotation\_range (40), width shift range (0.2), height shift range
(0.2), shear range (0.2), zoom range (0.2), and horizontal flip.

\hypertarget{custom-model---cnn-architecture}{%
\subsection{\texorpdfstring{\textbf{3.3 Custom Model - CNN
Architecture}}{3.3 Custom Model - CNN Architecture}}\label{custom-model---cnn-architecture}}

The custom model for this project consists of a Convolutional Neural
Network (CNN). CNN's are one of the most commonly used techniques in
Deep Learning for image recognition and classification. A CNN consists
of various types of layers, such as convolution layer, pooling layer,
fully-connected (FC) layer, batch normalization layer, and dropout
layer.

\begin{figure}[hbt!]
  \centering
  \includegraphics[width=6.5in,height=7.0in]{./Figure5.png}
  \caption{Showing a CNN Architecture for image classification\href{https://medium.com/techiepedia/binary-image-classifier-cnn-using-tensorflow-a3f5d6746697}{\underline{(Source)}}}
  \label{fig:figure5}
\end{figure}

As shown above, the CNN performs feature extraction. The early
convolution layers would extract simple features such as learning
directional derivatives (they look for edges in a particular direction).
The deeper convolution layers would learn more complex patterns other
than edges and direction, and as the layers go deeper and deeper, it
learns to identify even more complex patterns (higher-level features)
that the early layers are not able to detect, such as hard exudate, a
feature for classifying diabetic retinopathy.

The CNN Architecture for the custom model in this project consists of
various convolution layers with different filters at each stage, 3x3
kernel size, a stride of 1, and relu activation. After every two
convolution layers, a 2x2 max pooling layer was added along with a batch
normalization layer and a dropout layer of 0.4. After the final
convolution block, a flatten layer was added along with a few dense
layers and a dropout layer of 0.2 as a regularization technique to help
reduce overfitting. This network used softmax activation in the output
layer, Adam optimizer with a learning rate of 0.001, and categorical
cross-entropy as the loss function. More details on the custom CNN
architecture will be discussed later on in section 4.1 Training
Procedure.

\hypertarget{transfer-learning}{%
\subsection{\texorpdfstring{\textbf{3.4 Transfer
Learning}}{3.4 Transfer Learning}}\label{transfer-learning}}

Transfer learning is another popular technique in Deep Learning which
consists of using a pre-trained model on a new problem. This is
especially useful for reduced training time, increased performance, and
when there is less data available. Since the pre-trained models have
been trained on a large amount of data, they tend to generalize well on
new tasks and can be used on datasets such as the one for this project.

In this project, various transfer learning models were implemented such
as VGG16, ResNet50, and InceptionV3. These models will then be compared
to the custom CNN model during evaluation.

\hypertarget{vgg-16}{%
\subsection{\texorpdfstring{\textbf{3.4.1 VGG
16}}{3.4.1 VGG 16}}\label{vgg-16}}

VGG16 is a type of CNN that is one of the most popular models today. It
utilizes very small 3x3 convolution filters with a stride 1 and contains
approximately 138 trainable parameters. It is most commonly used for
object detection and classification, being able to achieve an average
accuracy of 92.7\%. The architecture of VGG16 is explained further
below.

\begin{figure}[hbt!]
  \centering
  \includegraphics[width=6.5in,height=7.0in]{./Figure6.png}
  \caption{Showing the VGG16 architecture}
  \label{fig:figure5}
\end{figure}

The VGG16 model contains 16 learnable parameter layers but a total of 21
layers (13 convolutional layers, 5 max-pooling layers, and 3 dense
layers). Each max pooling layer consists of a 2x2 filter with a stride
of 2. In Figure 6 above, the Conv 1 layers have 64 filters, the Conv 2
layers have 128 filters, the Conv 3 layers have 256 filters, and the
Conv 4 and Conv 5 layers have 512 filters.

\hypertarget{resnet}{%
\subsection{\texorpdfstring{\textbf{3.4.2
ResNet}}{3.4.2 ResNet}}\label{resnet}}

The ResNet model, which was introduced in 2015, introduced a concept
called Residual Network which solves the problem of vanishing/exploding
gradient. The vanishing gradient problem occurs when we increase the
number of layers which causes the gradient to become 0 or too large.
This gradient, in turn, increases the error while training and testing.
The technique of skip connections was introduced in this model, which
skips some layers and connects earlier layers to later layers hence
allowing information to bypass layers. This creates a residual block,
and a stack of residual blocks was used to create the ResNet model. The
purpose of this skip connection is so that if there are layers that will
hinder the performance of the model, those layers will be skipped during
regularization, which solves the problem of vanishing/exploding
gradient. An illustration of this architecture is shown below in Figure
7.

\begin{figure}[hbt!]
  \centering
  \includegraphics[width=6.5in,height=7.0in]{./Figure7.png}
  \caption{Showing the ResNet architecture}
  \label{fig:figure5}
\end{figure}

\hypertarget{inceptionv3}{%
\subsection{\texorpdfstring{\textbf{3.4.3
InceptionV3}}{3.4.3 InceptionV3}}\label{inceptionv3}}

The InceptionV3 is built from the older InceptionV1 (GoogLeNet)
architecture from 2014. Instead of using deep layers, the InceptionV1
architecture uses parallel layers to make it wider rather than deeper.
The motivation behind this is to avoid overfitting, which is caused by
using multiple deep layers. Four parallel layers are used in this
architecture: 1x1 convolution, 3x3 convolution, 5x5 convolution, and 3x3
max pooling

In addition, a 1x1 convolution layer was added before each of the above
layers to help overcome the high computational cost associated with this
architecture. The InceptionV3 takes this model a step further by
containing 42 layers and optimizes it using factorization into smaller
convolutions, spatial factorization into asymmetric convolutions,
utility of auxiliary classifiers, and efficient grid size reduction.

\begin{figure}[hbt!]
  \centering
  \includegraphics[width=6.5in,height=7.0in]{./Figure8.png}
  \caption{Showing the InceptionV3 architecture}
  \label{fig:figure5}
\end{figure}

\hypertarget{resultsdiscussion}{%
\subsection{\texorpdfstring{\textbf{4. Results/Discussion}}{Results/Discussion}}\label{resultsdiscussion}}

\hypertarget{training-procedure}{%
\subsection{\texorpdfstring{\textbf{4.1 Training
Procedure}}{4.1 Training Procedure}}\label{training-procedure}}

After preprocessing was completed, for training purposes, the training
set contained 4178 images, the validation set contained 2058 images, and
the testing set contained 1928 images. An initial learning rate of
0.001, Adam optimizer, and categorical crossentropy loss were used to
train the models for 20 epochs.

For model evaluation, various metrics were used, such as accuracy, loss,
precision, recall, F1 score, and AUC score. For this project, a total of
four unique models were built (a custom CNN model, a VGG16 model, a
ResNet50 model, and an InceptionV3 model). Each model was then trained
twice (once without data augmentation and once with data augmentation),
resulting in eight models for comparison.

The goal of performing data augmentation is to compare how each model
performs with and without data augmentation and to observe if any
improvements are made after training with data augmentation. For data
augmentation, various parameters were used, such as rotation, width
shift, height shift, shear, zoom, horizontal flip, and vertical flip.
Additionally, a rescale of 1./255 was used to normalize the pixels from
a range of 0-255 to 0-1.

When training using the custom CNN model, overfitting was observed after
10 epochs. To mitigate the effects of overfitting, various techniques
were implemented in the model. Such methods include adding an l2
regularizer (with a learning rate of 0.01) to some layers in the model,
adding weight he initialization to the conv layers, and adding dropout
layers (using a rate of 0.4 in the hidden layers and 0.2 for the output
layer). The effects of the mentioned regularization techniques can be
seen below.

\begin{figure}[hbt!]
  \centering
  \includegraphics[width=6.5in,height=7.0in]{./Figure9.png}
  \caption{Showing custom CNN model results before and after
regularization techniques were added to overcome overfitting.}
  \label{fig:figure5}
\end{figure}

\hypertarget{training-results}{%
\subsection{\texorpdfstring{\textbf{4.2 Training
Results}}{4.2 Training Results}}\label{training-results}}

\begin{figure}[hbt!]
  \centering
  \includegraphics[width=6.5in,height=7.0in]{./Figure10.png}
  \caption{Showing the training/validation loss and accuracy as a
function of epoch for each model trained}
  \label{fig:figure5}
\end{figure}

From Figure 10 above, it can be seen that the VGG16 model (without
augmentation) was only trained for 10 epochs while the others were
trained for 20 epochs. This is because after 10 epochs, the model
started to overfit, and at that point, it had already achieved
satisfactory accuracy and loss, hence, early stopping was performed. It
can also be observed that minimal overfitting is seen in the models
above that were trained without data augmentation.

On average, the training accuracy for the models without data
augmentation ranges from 70\%-80\%, but the validation accuracy in all
models is only able to reach 65\%. After researching and numerous
attempts at improving the model, it was found that due to the complexity
of the retina images in this dataset, further preprocessing is required
to achieve better results.

However, when looking at the models after data augmentation, we noticed
that some overfitting occurred, and the performance of each model
dropped. However, the training and validation accuracy is closer to each
other after performing data augmentation. This gives us a more realistic
idea as to how our model will perform on unseen noisy data. We can also
say that without data augmentation, the models may have been memorizing
the data rather than learning the patterns, which is why the training
accuracy was much higher than the validation accuracy on models without
data augmentation.

\hypertarget{evaluation-results}{%
\subsection{\texorpdfstring{\textbf{4.3 Evaluation
Results}}{4.3 Evaluation Results}}\label{evaluation-results}}

%\begin{longtable}[]{@{}llllllll@{}}
%\toprule
%Model & Accuracy & Loss & Precision & Recall & F1\_Score & AUC &
%Parameters\tabularnewline
%\midrule
%\endhead
%\begin{minipage}[t]{0.10\columnwidth}\raggedright
%Custom
%
%Model\strut
%\end{minipage} & \begin{minipage}[t]{0.10\columnwidth}\raggedright
%62\%\strut
%\end{minipage} & \begin{minipage}[t]{0.10\columnwidth}\raggedright
%1.17\strut
%\end{minipage} & \begin{minipage}[t]{0.10\columnwidth}\raggedright
%0.69\strut
%\end{minipage} & \begin{minipage}[t]{0.10\columnwidth}\raggedright
%0.54\strut
%\end{minipage} & \begin{minipage}[t]{0.10\columnwidth}\raggedright
%0.60\strut
%\end{minipage} & \begin{minipage}[t]{0.10\columnwidth}\raggedright
%0.88\strut
%\end{minipage} & \begin{minipage}[t]{0.10\columnwidth}\raggedright
%5,050,149\strut
%\end{minipage}\tabularnewline
%VGG16 & 66\% & 0.91 & 0.71 & 0.59 & 0.64 & 0.90 &
%14,789,205\tabularnewline
%ResNet50 & 65\% & 1.09 & 0.70 & 0.58 & 0.64 & 0.90 &
%25,826,693\tabularnewline
%InceptionV3 & 65\% & 1.29 & 0.70 & 0.60 & 0.65 & 0.90 &
%21,952,805\tabularnewline
%\begin{minipage}[t]{0.10\columnwidth}\raggedright
%Custom
%
%Model
%
%(augmented)\strut
%\end{minipage} & \begin{minipage}[t]{0.10\columnwidth}\raggedright
%55\%\strut
%\end{minipage} & \begin{minipage}[t]{0.10\columnwidth}\raggedright
%1.13\strut
%\end{minipage} & \begin{minipage}[t]{0.10\columnwidth}\raggedright
%0.90\strut
%\end{minipage} & \begin{minipage}[t]{0.10\columnwidth}\raggedright
%0.24\strut
%\end{minipage} & \begin{minipage}[t]{0.10\columnwidth}\raggedright
%0.38\strut
%\end{minipage} & \begin{minipage}[t]{0.10\columnwidth}\raggedright
%0.86\strut
%\end{minipage} & \begin{minipage}[t]{0.10\columnwidth}\raggedright
%5,050,149\strut
%\end{minipage}\tabularnewline
%\begin{minipage}[t]{0.10\columnwidth}\raggedright
%VGG16
%
%(augmented)\strut
%\end{minipage} & \begin{minipage}[t]{0.10\columnwidth}\raggedright
%61\%\strut
%\end{minipage} & \begin{minipage}[t]{0.10\columnwidth}\raggedright
%1.02\strut
%\end{minipage} & \begin{minipage}[t]{0.10\columnwidth}\raggedright
%0.67\strut
%\end{minipage} & \begin{minipage}[t]{0.10\columnwidth}\raggedright
%0.51\strut
%\end{minipage} & \begin{minipage}[t]{0.10\columnwidth}\raggedright
%0.58\strut
%\end{minipage} & \begin{minipage}[t]{0.10\columnwidth}\raggedright
%0.88\strut
%\end{minipage} & \begin{minipage}[t]{0.10\columnwidth}\raggedright
%14,789,205\strut
%\end{minipage}\tabularnewline
%\begin{minipage}[t]{0.10\columnwidth}\raggedright
%ResNet50
%
%(augmented)\strut
%\end{minipage} & \begin{minipage}[t]{0.10\columnwidth}\raggedright
%57\%\strut
%\end{minipage} & \begin{minipage}[t]{0.10\columnwidth}\raggedright
%1.23\strut
%\end{minipage} & \begin{minipage}[t]{0.10\columnwidth}\raggedright
%0.72\strut
%\end{minipage} & \begin{minipage}[t]{0.10\columnwidth}\raggedright
%0.37\strut
%\end{minipage} & \begin{minipage}[t]{0.10\columnwidth}\raggedright
%0.49\strut
%\end{minipage} & \begin{minipage}[t]{0.10\columnwidth}\raggedright
%0.86\strut
%\end{minipage} & \begin{minipage}[t]{0.10\columnwidth}\raggedright
%25,826,693\strut
%\end{minipage}\tabularnewline
%\begin{minipage}[t]{0.10\columnwidth}\raggedright
%InceptionV3
%
%(augmented)\strut
%\end{minipage} & \begin{minipage}[t]{0.10\columnwidth}\raggedright
%58\%\strut
%\end{minipage} & \begin{minipage}[t]{0.10\columnwidth}\raggedright
%1.21\strut
%\end{minipage} & \begin{minipage}[t]{0.10\columnwidth}\raggedright
%0.73\strut
%\end{minipage} & \begin{minipage}[t]{0.10\columnwidth}\raggedright
%0.39\strut
%\end{minipage} & \begin{minipage}[t]{0.10\columnwidth}\raggedright
%0.50\strut
%\end{minipage} & \begin{minipage}[t]{0.10\columnwidth}\raggedright
%0.87\strut
%\end{minipage} & \begin{minipage}[t]{0.10\columnwidth}\raggedright
%21,952,805\strut
%\end{minipage}\tabularnewline
%\bottomrule
%\end{longtable}

\begin{figure}[hbt!]
  \centering
  \includegraphics[width=6.5in,height=7.0in]{./Figure11.png}
  \caption{Showing the evaluation of each model on validation data}
  \label{fig:figure5}
\end{figure}

The models could not be evaluated on test data. Since this dataset
belongs to a Kaggle competition, the goal is to label the test data
while not being provided with the true labels. Therefore, the models
were evaluated on validation data.

From Figure 11 above, it can be seen that the VGG-16 and ResNet50 models
(without data augmentation) perform the best when evaluated on
validation data. They were both able to achieve a validation accuracy of
66\% and a validation loss of 0.91 and 0.97, respectively. The other two
models still do a very good job, and their performance is very close to
that of VGG-16 and ResNet50.

When looking at the model results after performing data augmentation, we
notice a decrease in performance throughout all four models. A possible
reason for this can be that due to the possibility that further
preprocessing of the image could have been done to increase robustness
(based on knowledge in the domain), the quality of images augmented
would not have been very robust hence resulting in lower performing
models after data augmentation.

\hypertarget{labelling-test-data}{%
\subsection{\texorpdfstring{\textbf{4.4 Labelling Test
Data}}{4.4 Labelling Test Data}}\label{labelling-test-data}}

The final goal of this project is to label the test images. For this,
the model.predict() function was used to predict each image in the test
dataset using each of the eight (8) models. The labels of each model
were then appended to the test.csv file hence successfully labeling the
test data using all models. This labeled file can be found within the
project GitHub repository. There is no way to know if the predictions
are correct since the true label for the test data was not provided in
the Kaggle competition.

\hypertarget{conclusion}{%
\subsection{\texorpdfstring{\textbf{5. Conclusion}}{Conclusion}}\label{conclusion}}

In conclusion, our custom CNN model was able to perform relatively well
when compared to popular Transfer Learning models such as VGG16,
ResNet50, and InceptionV3. Adding data augmentation technique did not
improve model performance, and a potential reason for this can be due to
noisy images and insufficient image preprocessing because of a lack of
domain knowledge. Various regularization techniques such as l2
regularizer, weight he initializer, and dropout layers helped to
mitigate the effects of overfitting.

In the future, the team plans to conduct more research and study this
domain properly to better preprocess the retina images. This will enable
the data to be more robust, resulting in better model performance.
Additionally, due to time constraints, ET Decision Trees was not able to
be done, so this is something that will be looked into further. Lastly,
our group plans to extend this project by developing a model that uses
retinal fundus images to predict the age at which a patient on an
unhealthy dietary plan is likely to suffer from blindness due to
retinopathy.

\hypertarget{github-link-for-project}{%
\subsection{\texorpdfstring{\textbf{6. GitHub Link for
Project}}{GitHub Link for Project}}\label{github-link-for-project}}

Project code and data are available at

\href{https://github.com/Girish-Rajani/Blindness_Detection_Deep_Learning}{\underline{https://github.com/Girish-Rajani/Blindness\_Detection\_Deep\_Learning}}

\hypertarget{contributions}{%
\subsection{\texorpdfstring{\textbf{7. Contributions}}{Contributions}}\label{contributions}}

Girish Rajani Bathija

\begin{itemize}
\item
  \begin{quote}
  Built custom CNN model and VGG-16 model (with and without data
  augmentation)
  \end{quote}
\item
  \begin{quote}
  Performed Hyperparameter tuning and added regularization techniques
  \end{quote}
\item
  \begin{quote}
  Training and Evaluation and Labelled Test data
  \end{quote}
\item
  \begin{quote}
  Worked on the report
  \end{quote}
\end{itemize}

Shriya Prasanna

\begin{itemize}
\item
  \begin{quote}
  Performed Image Resizing
  \end{quote}
\end{itemize}

\begin{itemize}
\item
  \begin{quote}
  Built ResNet50 model and InceptionV3 model (with and without data
  augmentation)
  \end{quote}
\item
  \begin{quote}
  Training and Evaluation
  \end{quote}
\item
  \begin{quote}
  Worked on the report
  \end{quote}
\end{itemize}

Bhavesh Rajesh Talreja

\begin{itemize}
\item
  \begin{quote}
  Performed Train/Validation Split
  \end{quote}
\end{itemize}

\begin{itemize}
\item
  \begin{quote}
  Performed Image Augmentation on imbalanced classes
  \end{quote}
\item
  \begin{quote}
  Built ResNet50 model and InceptionV3 model (with and without data
  augmentation)
  \end{quote}
\item
  \begin{quote}
  Worked on LaTeX conversion of the report
  \end{quote}
\end{itemize}

\hypertarget{references}{%
\subsection{\texorpdfstring{\textbf{8. References}}{References}}\label{references}}

\textbf{Dataset:}

\href{https://www.kaggle.com/competitions/aptos2019-blindness-detection/data}{\underline{https://www.kaggle.com/competitions/aptos2019-blindness-detection/data}}

{[}1{]} https://arxiv.org/ftp/arxiv/papers/2006/2006.07475.pdf

{[}2{]} Lam C, Yi D, Guo M, Lindsey T. Automated Detection of Diabetic
Retinopathy using Deep Learning. AMIA Jt Summits Transl Sci Proc. 2018
May 18;2017:147-155. PMID: 29888061; PMCID: PMC5961805.

{[}3{]} Ramachandran, N., Hong, S.C., Sime, M.J. and Wilson, G.A.
(2018), Diabetic retinopathy screening using deep neural network. Clin.
Experiment. Ophthalmol., 46: 412-416.
\href{https://doi.org/10.1111/ceo.13056}{\underline{https://doi.org/10.1111/ceo.13056}}.

{[}4{]} Sreng, S.; Maneerat, N.; Hamamoto, K.; Panjaphongse, R.
Automated Diabetic Retinopathy Screening System Using Hybrid Simulated
Annealing and Ensemble Bagging Classifier. Appl. Sci. 2018, 8, 1198.
\href{https://doi.org/10.3390/app8071198}{\underline{https://doi.org/10.3390/app8071198}}.

{[}5{]}https://github.com/adityasurana/APTOS-Blindness-Detection-Kaggle/blob/master/APTOS\_Blind\_Detection.ipynb

{[}6{]} https://iq.opengenus.org/inception-v3-model-architecture/

{[}7{]}
https://www.geeksforgeeks.org/residual-networks-resnet-deep-learning/\#

{[}8{]}https://medium.com/@mygreatlearning/everything-you-need-to-know-about-vgg16-7315defb5918

{[}9{]} https://www.upgrad.com/blog/basic-cnn-architecture/

{[}10{]}https://www.analyticsvidhya.com/blog/2020/09/overfitting-in-cnn-show-to-treat-overfitting-in-convolutional-neural-networks/

{[}11{]}https://aakashgoel12.medium.com/how-to-add-user-defined-function-get-f1-score-in-keras-metrics-3013f979ce0d

\end{document}
